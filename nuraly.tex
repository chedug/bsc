%%%%%%%%%%%%%%%%%%%%%%%%%%%%%%%%%%%%%%%%%%%%%%
%                                            %
%    Dyussenov Nuraly BSc Thesis Work        %
%                                            %
%%%%%%%%%%%%%%%%%%%%%%%%%%%%%%%%%%%%%%%%%%%%%%

\documentclass[12pt,a4paper,oneside]{book} % twoside,openany

% Packages
\usepackage[T1]{fontenc}
\usepackage[utf8]{inputenc}

\usepackage{amsmath,amssymb,amsthm}

\usepackage{enumerate}
\usepackage{graphicx}
\usepackage{caption,subcaption}
\graphicspath{{./figures/}}

\usepackage{array}
\newcolumntype{L}[1]{>{\raggedright\let\newline\\\arraybackslash\hspace{0pt}}m{#1}}
\newcolumntype{C}[1]{>{\centering\let\newline\\\arraybackslash\hspace{0pt}}m{#1}}
\newcolumntype{R}[1]{>{\raggedleft\let\newline\\\arraybackslash\hspace{0pt}}m{#1}}

\usepackage{url}
\usepackage{hyperref}
\usepackage{refcheck}
\usepackage{tikz}

% Theoremlike environment
\newtheorem{theorem}{Theorem}[section]
\newtheorem{proposition}[theorem]{Proposition}
\newtheorem{lemma}[theorem]{Lemma}
\newtheorem{corollary}[theorem]{Corollary}
\newtheorem{remark}[theorem]{Remark}
\newtheorem{definition}[theorem]{Definition}
\newtheorem{example}[theorem]{Example}
\newtheorem{assumption}[theorem]{Assumption}

% Ususal abbreviations
\newcommand{\N}{\mathbb{N}}
\newcommand{\Z}{\mathbb{Z}}
\newcommand{\Q}{\mathbb{Q}}
\newcommand{\R}{\mathbb{R}}

% Probability theory
\newcommand{\A}{\mathcal{A}}
\newcommand{\B}{\mathcal{B}}
\newcommand{\C}{\mathcal{C}}
\newcommand{\F}{\mathcal{F}}
\newcommand{\G}{\mathcal{G}}
\newcommand{\X}{\mathcal{X}}
\newcommand{\Y}{\mathcal{Y}}
\newcommand{\M}{\mathcal{M}}

\renewcommand{\P}{\mathbb{P}}
\newcommand{\E}{\mathbb{E}}
\newcommand{\D}{\mathbb{D}}
\newcommand{\law}[1]{\text{Law}(#1)}

\newcommand{\Pas}{\text{a.s.}}
\newcommand{\ind}{\mathds{1}}

% Analysis
\newcommand{\eps}{\varepsilon}
\newcommand{\la}{\lambda}
\newcommand{\ga}{\gamma}
\newcommand{\ka}{\kappa}
\newcommand{\dtv}{d_{\text{TV}}}

% Integration
\newcommand{\dint}{\mathrm{d}} 

\newcommand{\lfrf}[1]{\lfloor #1\rfloor}


%%%%%%%%%%%%%%%%%%%%%%%%%%%%%%%%%%%%55% Page layout
\usepackage{indentfirst}
%\usepackage{fullpage}
\usepackage[a4paper]{geometry}
\geometry{tmargin=3cm,lmargin=3.5cm,rmargin=2cm}
% manual page formatting
%\setlength{\headsep}{25pt}
\hyphenation{}

% headers and footers
\usepackage{fancyhdr}
\usepackage{mathptmx}

\newcommand\HRule{\rule{\textwidth}{1pt}}

\usepackage{varwidth}

%\usepackage{booktabs}
\usepackage{multirow,array}
\usepackage{siunitx}

% Quotations
\usepackage{csquotes}

\makeatletter
\renewcommand{\@chapapp}{}% Not necessary...
\newenvironment{chapquote}[2][2em]
{\setlength{\@tempdima}{#1}%
	\def\chapquote@author{#2}%
	\parshape 1 \@tempdima \dimexpr\textwidth-2\@tempdima\relax%
	\itshape}
{\par\normalfont\hfill--\ \chapquote@author\hspace*{\@tempdima}\par\bigskip}
\makeatother


\usepackage{titlesec, blindtext, color}
\definecolor{gray75}{gray}{0.75}
\newcommand{\hsp}{\hspace{20pt}}


\newcommand{\stdwidth}{0.61\linewidth}

% Space above chapter titles
\usepackage{titlesec}

\renewcommand{\baselinestretch}{1.5}



\begin{document}

\begin{titlepage}
	
	%\newcommand{\HRule}{\rule{\linewidth}{0.5mm}} % Defines a new command for the horizontal lines, change thickness here
	
	\center % Center everything on the page
	
	%----------------------------------------------------------------------------------------
	%	HEADING SECTIONS
	%----------------------------------------------------------------------------------------
	
	\textsc{\LARGE Budapesti University of Technology and Economics}\\[1.5cm] % Name of your university/college
	\textsc{\Large Institute of Mathematics}\\[0.5cm] % Major heading such as course name
	\textsc{\large Faculty of Mathematics}\\[0.5cm] % Minor heading such as course title
	
	%----------------------------------------------------------------------------------------
	%	TITLE SECTION
	%----------------------------------------------------------------------------------------
	
	\HRule \\[0.4cm]
	{ \Large \bfseries Linear Regression through Origin
		 }\\[0.4cm]
	\HRule \\[1.5cm]
	
	%----------------------------------------------------------------------------------------
	%	AUTHOR SECTION
	%----------------------------------------------------------------------------------------
	
	\begin{tabular}{L{6cm} R{8cm}}
	\emph{Author:}   & \emph{Supervisor:} \\
	Dyussenov Nuraly & Dr. Jozsef Mala   \\
	                 & Associate Professor, BME Fac. of Nat. Sci. 
	\end{tabular}\\[1.3cm]

	
	\vfill
	{\Large Budapest, \today}\\[1.2cm] % Date, change the \today to a set date if you want to be precise
		
	\includegraphics[trim={0cm 0cm 0cm 0cm},clip,width=0.5\linewidth]{bme_logo_nagy.eps}
	 % Include a department/university logo - this will require the graphicx package
	
	%\vfill % Fill the rest of the page with whitespace
	
\end{titlepage}

	%\newpage\null\thispagestyle{empty}\newpage

	\frontmatter
	%\chapter*{\centering Kivonat}
	
	\tableofcontents
	\listoftables
	\listoffigures
		
	\mainmatter
	
	\fancypagestyle{plain}{%

		\fancyhf{}
		\fancyhead[L]{\rule[-2ex]{0pt}{2ex}\small \leftmark} 
		\fancyhead[R]{} 
		\fancyfoot[L]{}
		\fancyfoot[C]{-- \thepage\ --}
		\fancyfoot[R]{} 
		\renewcommand{\headrulewidth}{1.5pt}
		\renewcommand{\footrulewidth}{1pt}}
	\pagestyle{plain}
	
	\titleformat{\chapter}[display]{\normalfont\huge\bfseries}{\chaptertitlename\ \thechapter}{20pt}{\Huge}
	\titlespacing*{\chapter}{10pt}{20pt}{40pt}
	 
	\titleformat{\chapter}[hang]{\Huge\bfseries}{\thechapter.\hsp}{0pt}{\Huge\bfseries} 
	  
	\chapter{Introduction} % ~2-3 pages
	
	\begin{chapquote}{XY} % Quotation (optional)
	''Bla-bla-bla''
	\end{chapquote}


	\chapter{Theoretical background} % ~10 pages (by Dec 31)
	% Every chapter should start with a short summary

	\section{Statistics Basics}
	% Give definitions of Smooth manifold, differentiable maps between manifolds, diffeomorphisms, vectorfields, differential equations and flows, riemannian structure

	\section{Simple Linear Regression}
	% Definitions of dynamical systems, autonomous differential equations, flows, attractors, and topological properties, invariant like Lyapunov exponent, etc. With examples...

	\section{Simple Linear Regression with no intercept term}
	% State these theorems, the connection between them and their meanings, historical background, Takens' original motivations
	
	\section{To add}	
	% Defnition, examples, representation via iterated random functions, analogy with dynamical systems, Rényi information dimension of a point cloud
	
	
	\chapter{Applications to Linear Regression through Origin} % (By Jan 31)

	\section{Something to add 1} 
	% UNICORNS, Temporary Outlier Factor
	% Outlier types (point, trend and contextual outliers)
	
	\section{Something to add 1} 
	% Sugihara's algorithm (CCM) and the new entropy-based methods

	\chapter{Theoretical results} % (By Feb 26)
	
	\section{A theoretical resilt}
	
	\section{Towards some advanced topic}
	% Why is it desirable to establish such result? Do a short literature review. State a conjecture on the potential form of Taken's embedding theorem. Try to prove it (or at least test it) for autoregressive processes.

	\chapter{Programming simulations} % (By March 31)
	% Demonstrate the applicability of these algorithms on interesting data sets
	% Implementation details, how to find optimal time lag, and embedding dimension


		
	\chapter{Summary and closing words}\label{ch:closing}
	% 1 page
	



	% All in all 26-30 pages
	
	% IRODALOMJEGYZÉK
	\bibliography{nuraly}
	\bibliographystyle{plain}
	
	\appendix
	
	\chapter{Program Codes}\label{ap:codes}
	
\end{document}
