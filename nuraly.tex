%%%%%%%%%%%%%%%%%%%%%%%%%%%%%%%%%%%%%%%%%%%%%%
%                                            %
%    Dyussenov Nuraly BSc Thesis Work        %
%                                            %
%%%%%%%%%%%%%%%%%%%%%%%%%%%%%%%%%%%%%%%%%%%%%%

\documentclass[12pt,a4paper,oneside]{book} % twoside,openany

% Packages
\usepackage[T1]{fontenc}
\usepackage[utf8]{inputenc}

\usepackage{amsmath,amssymb,amsthm}

\usepackage{enumerate}
\usepackage{graphicx}
\usepackage{caption,subcaption}
\graphicspath{{./figures/}}

\usepackage{array}
\newcolumntype{L}[1]{>{\raggedright\let\newline\\\arraybackslash\hspace{0pt}}m{#1}}
\newcolumntype{C}[1]{>{\centering\let\newline\\\arraybackslash\hspace{0pt}}m{#1}}
\newcolumntype{R}[1]{>{\raggedleft\let\newline\\\arraybackslash\hspace{0pt}}m{#1}}

\usepackage{url}
\usepackage{hyperref}
\usepackage{refcheck}
\usepackage{tikz}

% Theoremlike environment
\newtheorem{theorem}{Theorem}[section]
\newtheorem{proposition}[theorem]{Proposition}
\newtheorem{lemma}[theorem]{Lemma}
\newtheorem{corollary}[theorem]{Corollary}
\newtheorem{remark}[theorem]{Remark}
\newtheorem{definition}[theorem]{Definition}
\newtheorem{example}[theorem]{Example}
\newtheorem{assumption}[theorem]{Assumption}

% Ususal abbreviations
\newcommand{\N}{\mathbb{N}}
\newcommand{\Z}{\mathbb{Z}}
\newcommand{\Q}{\mathbb{Q}}
\newcommand{\R}{\mathbb{R}}

% Probability theory
\newcommand{\A}{\mathcal{A}}
\newcommand{\B}{\mathcal{B}}
\newcommand{\C}{\mathcal{C}}
\newcommand{\F}{\mathcal{F}}
\newcommand{\G}{\mathcal{G}}
\newcommand{\X}{\mathcal{X}}
\newcommand{\Y}{\mathcal{Y}}
\newcommand{\M}{\mathcal{M}}

\renewcommand{\P}{\mathbb{P}}
\newcommand{\E}{\mathbb{E}}
\newcommand{\D}{\mathbb{D}}
\newcommand{\law}[1]{\text{Law}(#1)}

\newcommand{\Pas}{\text{a.s.}}
\newcommand{\ind}{\mathds{1}}

% Analysis
\newcommand{\eps}{\varepsilon}
\newcommand{\la}{\lambda}
\newcommand{\ga}{\gamma}
\newcommand{\ka}{\kappa}
\newcommand{\dtv}{d_{\text{TV}}}

% Integration
\newcommand{\dint}{\mathrm{d}} 

\newcommand{\lfrf}[1]{\lfloor #1\rfloor}


%%%%%%%%%%%%%%%%%%%%%%%%%%%%%%%%%%%%55% Page layout
\usepackage{indentfirst}
%\usepackage{fullpage}
\usepackage[a4paper]{geometry}
\geometry{tmargin=3cm,lmargin=3.5cm,rmargin=2cm}
% manual page formatting
%\setlength{\headsep}{25pt}
\hyphenation{}

% headers and footers
\usepackage{fancyhdr}
\usepackage{mathptmx}

\newcommand\HRule{\rule{\textwidth}{1pt}}

\usepackage{varwidth}

%\usepackage{booktabs}
\usepackage{multirow,array}
\usepackage{siunitx}

% Quotations
\usepackage{csquotes}

\makeatletter
\renewcommand{\@chapapp}{}% Not necessary...
\newenvironment{chapquote}[2][2em]
{\setlength{\@tempdima}{#1}%
	\def\chapquote@author{#2}%
	\parshape 1 \@tempdima \dimexpr\textwidth-2\@tempdima\relax%
	\itshape}
{\par\normalfont\hfill--\ \chapquote@author\hspace*{\@tempdima}\par\bigskip}
\makeatother


\usepackage{titlesec, blindtext, color}
\definecolor{gray75}{gray}{0.75}
\newcommand{\hsp}{\hspace{20pt}}


\newcommand{\stdwidth}{0.61\linewidth}

% Space above chapter titles
\usepackage{titlesec}

\renewcommand{\baselinestretch}{1.5}



\begin{document}

\begin{titlepage}
	
	%\newcommand{\HRule}{\rule{\linewidth}{0.5mm}} % Defines a new command for the horizontal lines, change thickness here
	
	\center % Center everything on the page
	
	%----------------------------------------------------------------------------------------
	%	HEADING SECTIONS
	%----------------------------------------------------------------------------------------
	
	\textsc{\LARGE Budapesti University of Technology and Economics}\\[1.5cm] % Name of your university/college
	\textsc{\Large Institute of Mathematics}\\[0.5cm] % Major heading such as course name
	\textsc{\large Faculty of Mathematics}\\[0.5cm] % Minor heading such as course title
	
	%----------------------------------------------------------------------------------------
	%	TITLE SECTION
	%----------------------------------------------------------------------------------------
	
	\HRule \\[0.4cm]
	{ \Large \bfseries Linear Regression through Origin
		 }\\[0.4cm]
	\HRule \\[1.5cm]
	
	%----------------------------------------------------------------------------------------
	%	AUTHOR SECTION
	%----------------------------------------------------------------------------------------
	
	\begin{tabular}{L{6cm} R{8cm}}
	\emph{Author:}   & \emph{Supervisor:} \\
	Dyussenov Nuraly & Dr. Jozsef Mala   \\
	                 & Associate Professor, BME Fac. of Nat. Sci. 
	\end{tabular}\\[1.3cm]

	
	\vfill
	{\Large Budapest, \today}\\[1.2cm] % Date, change the \today to a set date if you want to be precise
		
	\includegraphics[trim={0cm 0cm 0cm 0cm},clip,width=0.5\linewidth]{bme_logo_nagy.eps}
	 % Include a department/university logo - this will require the graphicx package
	
	%\vfill % Fill the rest of the page with whitespace
	
\end{titlepage}

	%\newpage\null\thispagestyle{empty}\newpage

	\frontmatter
	%\chapter*{\centering Kivonat}
	
	\tableofcontents
	\listoftables
	\listoffigures
		
	\mainmatter
	
	\fancypagestyle{plain}{%

		\fancyhf{}
		\fancyhead[L]{\rule[-2ex]{0pt}{2ex}\small \leftmark} 
		\fancyhead[R]{} 
		\fancyfoot[L]{}
		\fancyfoot[C]{-- \thepage\ --}
		\fancyfoot[R]{} 
		\renewcommand{\headrulewidth}{1.5pt}
		\renewcommand{\footrulewidth}{1pt}}
	\pagestyle{plain}
	
	\titleformat{\chapter}[display]{\normalfont\huge\bfseries}{\chaptertitlename\ \thechapter}{20pt}{\Huge}
	\titlespacing*{\chapter}{10pt}{20pt}{40pt}
	 
	\titleformat{\chapter}[hang]{\Huge\bfseries}{\thechapter.\hsp}{0pt}{\Huge\bfseries} 
	  
	\chapter{Introduction} % ~2-3 pages
	
	\begin{chapquote}{XY} % Quotation (optional)
	''Bla-bla-bla''
	\end{chapquote}


	\chapter{Theoretical background} % ~10 pages (by Dec 31)
	% Every chapter should start with a short summary


	\section{Statistics Basics}
	% Give definitions of mean, variance, statistic, estimators, sufficiency, bias, MSE, and etc.

\textbf{Definition (Data)} Let $(x_1, \ldots, x_n)$, where $x_i \in S$ for $i = 1, \ldots, n$. The set $S$ is typically $\mathbb{R}$, $\mathbb{R}^d$, or it can be any abstract set. However, for our purposes, $S$ (the sample space) will usually be $\mathbb{R}$.

\textbf{Definition (Sample)} In statistics, our data are often modeled by a vector $\mathbf{X} = (X_1, X_2, \ldots, X_n)$ of i.i.d. (independent, identically distributed) random variables, called the sample (of which size is $n$), where the random variables $X_i$ take values in $\mathbb{Z}$ or $\mathbb{R}$. The common distribution of the $X_i$ is called the parent distribution, and we say that the sample is from that parent distribution.

\textbf{Definition (Model)} A statistical model is a family $\{P_{\theta} \,|\, \theta \in \Theta\}$ of distributions on the sample space. When $\Theta \subset \mathbb{R}^d$, we say that we have a parametric model, and we call $\Theta$ the parameter set (space).

\textbf{Definition (p-th Quantile of Data)} If $p \in (0, 1)$, then a $p$-th quantile (or a $p$-th percentile) of the data $(x_1, \ldots, x_n)$ is a $p$-th quantile of the corresponding empirical distribution function $\hat{F}_n$.

\textbf{Definition (Sample mean)} Let $(X_1, \ldots, X_n)$ be a sample. Then the random variable
\[ \bar{X} = X = \frac{1}{n} \sum_{i=1}^{n} X_i \]
is called the sample mean.

\textbf{Definition (Estimator)} An estimator is a statistic (a function of the sample data) used to estimate an unknown parameter in a statistical model. An estimator for the parameter $\theta$, denoted as $\hat{\theta}$, is any measurable function of the random variables $X_1, X_2, \ldots, X_n$.

\textbf{Definition (Biased)} If $\hat{\theta}$ is an estimator of $\theta$, then we can define the quantity \textit{Bias}($\hat{\theta}$) = $\mathbb{E}_{\theta}[\hat{\theta}] - \theta$. The estimator $\hat{\theta}$ is called unbiased if its bias is 0.

\textbf{Definition (MSE of an Estimator)} Let us have the model $\{P_{\theta} \,|\, \theta \in \Theta\}$ and let us have the sample $(X_1, \ldots, X_n)$ from it. The mean square error (or the quadratic risk) of an estimator $\hat{\theta} = \hat{\theta}(X_1, \ldots, X_n)$ for the parameter $\theta$ is defined by
\[ \text{MSE}_{\theta}(\hat{\theta}) = \mathbb{E}_{\theta}((\hat{\theta} - \theta)^2) \]
when $\theta$ is the true parameter.

\textbf{Steiner's identity:} $\mathbb{E}((X - a)^2) = \text{Var}(X) + (a - \mathbb{E}(X))^2$

\textbf{Interpretation in the context of mean square error (MSE):}

\[ \text{MSE}_{\theta}(\hat{\theta}) = \text{Var}_{\theta}(\hat{\theta}) + (\text{Bias}_{\theta}(\hat{\theta}))^2 \]

\textbf{Definition (Sufficiency)} Let the model be $\{P_{\theta} \,|\, \theta \in \Theta\}$ and $\mathbf{X} = (X_1, \ldots, X_n)$ be a sample from it. The statistic $T$ is called \textit{sufficient} for the parameter $\theta$ (or, for the model $\{P_{\theta} \,|\, \theta \in \Theta\}$) if the conditional distribution $P_{\theta}(\mathbf{X} \in \cdot \,|\, T = t)$ does not depend on $\theta$.

\textbf{Theorem (Neyman-Fisher Factorization Theorem)} If the model is $\{p(x|\theta) \,|\, \theta \in \Theta\}$ where $p(x|\theta)$ is a probability mass/density function and $\mathbf{X} = (X_1, \ldots, X_n)$ is a sample from it, then the statistic $T$ is \textit{sufficient} for the parameter $\theta$ if and only if we can find nonnegative functions $g$ and $h$ such that
\[ p_{\mathbf{X}}(x | \theta) = g(T(x), \theta)h(x). \]

\textbf{Definition (Likelihood)} Let $\{p(x, \theta), \theta \in \Theta\}$ be a model. If the observed value of $X$ is $x$, we say that $p(x | \theta)$ is the \textit{likelihood} of $\theta$: $L(\theta) = p(x | \theta)$. Thus, we are considering the mass/density as a function of $\theta$, for a fixed $x$. If $x = (x_1, \ldots, x_n)$ is a realization of the sample $\mathbf{X} = (X_1, \ldots, X_n)$, then $p(x | \theta)$ is the product of the marginals,
\[ L(\theta) = p(x | \theta) = \prod_{i=1}^{n} p(x_i | \theta). \]




	\clearpage
	
	\section{Simple Linear Regression}
	% Definitions of SLR

	\clearpage

	\section{Simple Linear Regression with no intercept term}
	% State these theorems, the connection between them and their meanings, historical background
	
	\clearpage

	\section{Comparative Analysis}	
	% If applicable, compare Linear Regression through Origin with other regression techniques. Discuss why this particular method was chosen over others, highlighting its advantages and limitations in comparison to traditional linear regression models.
	
	\clearpage

	
	\chapter{Applications to Linear Regression through Origin} % (By Nov 31)


	\clearpage

	\section{Something to add 1} 
	% UNICORNS, Temporary Outlier Factor
	% Outlier types (point, trend and contextual outliers)
	
	\section{Something to add 1} 
	% Sugihara's algorithm (CCM) and the new entropy-based methods

	\chapter{Theoretical results} % (By Feb 26)
	
	\section{A theoretical resilt}
	
	\section{Towards some advanced topic}
	% Why is it desirable to establish such result? Do a short literature review. State a conjecture on the potential form of Taken's embedding theorem. Try to prove it (or at least test it) for autoregressive processes.

	\chapter{Programming simulations} % (By March 31)
	% Demonstrate the applicability of these algorithms on interesting data sets
	% Implementation details, how to find optimal time lag, and embedding dimension


		
	\chapter{Summary and closing words}\label{ch:closing}
	% 1 page
	



	% All in all 26-30 pages
	
	% IRODALOMJEGYZÉK
	\bibliography{nuraly}
	\bibliographystyle{plain}
	
	\appendix
	
	\chapter{Program Codes}\label{ap:codes}
	
\end{document}
